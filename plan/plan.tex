\documentclass{article}

\begin{document}

\section{Bikes}
Several bicycles will be measured to get a nice spread of property data from different bicycles. These are the possible choices:
\begin{itemize}
	\item Batavus Browser
	\item Batavus Browser with DAQ equipment
	\item Batavus Stratos Deluxe
	\item Batavus Crescendo with solid fork
	\item Yellow bike
	\item Bianchi Pista
	\item Jodi's mountain bike
	\item Miriam's road bike
\end{itemize}

\section{Geometry}
\begin{description}
	\item[$w$ wheelbase] fix the steer angle to zero by tying the wheel to the downtube. Using a ruler measure from the center of the front wheel to the center of the rear wheel.
	\item[$r_\mathrm{F}$ front wheel radius] Pump the tires to the recommended inflation pressure. Mark a fine line on the tire at the ground contact point (Use a square to find ground contact). Roll the bicycle through as many full revolutions that can be measured with a single measuring tape. Do this with and without a rider. Note the mass of the rider, revolutions and the distance. $r_\mathrm{F}=\frac{d}{2\pi n}$
	\item[$r_\mathrm{R}$ rear wheel radius] Same as the front wheel.
	\item[$\lambda$ steer axis angle] fix the bicycle such that the roll angle and steer angle are both zero. Affix a straightedge to the head tube such that extends to the ground (possibly design a universal attachment that takes the guessing out). Use the digital angle gage to measure the angle between the floor and the straightedge. $\lambda=\frac{\pi}{2}-\lambda_{ht}$
	\item[$c$ trail] After measuring the head tube angle, use a square to locate the ground contact point relative to the center of the wheel. Measure the distance from the ground contact to the intersection of the headtube straightedge and the ground. Also, remove the fork from the bicycle and mount it horizontally by the steer tube on a table. Measure the fork offset using a height gage.
\end{description}

\section{Mass}
Measure the mass of the entire bicycle by hanging it on a hook scale. Measure each individual part using a tabletop scale.

\section{Center of Mass}


\section{Moment of inertia}
\subsection{Torsional Pendulum}
The torsional pendulum will be calibrated with a long rod of known inertia each time the pendulum is set up to determine the coefficient of elasticity of the rod/fixture.
\subsection{Frame}
\end{document}