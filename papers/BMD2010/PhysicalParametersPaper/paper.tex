\documentclass{bmd2010p}

%\usepackage[colorlinks=true, linkcolor=blue]{hyperref}
\usepackage{graphicx}
\usepackage{subfiles}
\usepackage{subfig}
\usepackage{amsmath}
\usepackage{pdflscape}

\begin{document}

% title
\begin{center}
\fontsize{14}{20}{\bf Accurate Measurement of Bicycle Parameters}
\end{center}

%%%%%%%%%%%%%%%% authors %%%%%%%%%%%%%%%
\begin{center}
\normalsize{\bf{Jason K. Moore$^{*}$, Mont Hubbard$^{*}$, A. L. Schwab$^\dag$,
            J. D. G. Kooijman$^\dag$}}
\end{center}

\begin{center}
\begin{tabular}{c}
$^*$ Department of Mechanical and Aerospace Engineering\\
University of California, Davis\\
One Shields Avenue, Davis, CA, 95616, USA\\
e-mail: jkmoor@ucdavis.edu, mhubbard@ucdavis.edu\\
\end{tabular}
\begin{tabular}{c}
$^\dag$ Laboratory for Engineering Mechanics\\
Delft University of Technology\\
Mekelweg 2, 2628CD Delft, The Netherlands\\
e-mail: a.l.schwab@tudelft.nl, jodikooijman@gmail.com\\
\end{tabular} \\ \vspace{2.5ex}
\end{center}
\section*{ABSTRACT}
Accurate measurements of a bicycle's physical parameters are required for
realistic dynamic simulations and analysis. The most basic models require the
geometry, mass, mass location and mass distributions for the rigid bodies. More
complex models require estimates of tire characteristics, human
characteristics, friction, stiffness, damping, etc. In this
paper we present the measurement of the minimal bicycle parameters required for
the benchmark Whipple bicycle model presented in~\cite{Meijaard2007}. This
model is composed of four rigid bodies, has ideal rolling and frictionless joints,
and is laterally symmetric. A set of 25 parameters describes the
geometry, mass, mass location and mass distribution of each of the rigid
bodies. The experimental methods used to estimate the parameters described
herein are based primarily on the work done in~\cite{Kooijman2006} but have
been refined for improved accuracy and methodology. Koojiman's work was
preceded by \cite{Roland1971} who measured a bicycle in a
similar fashion and both~\cite{Dohring1953} and~\cite{Singh1971} who used
similar techniques with scooters.

We measured the physical characteristics of six different bicycles, two of
which were set up in two different configurations. The six bicycles, chosen for
both variety and convenience, are as follows:
\emph{Batavus Browser}, a Dutch style city bicycle measured with and without
instrumentation as described in~\cite{Kooijman2009a}; \emph{Batavus Stratos
Deluxe}, a Dutch style sporty city bicycle; \emph{Batavus Crescendo Deluxe} a
Dutch style city bicycle with a suspended fork; \emph{Gary Fisher Mountain
Bike}, a hardtail mountain bicycle; \emph{Bianchi Pista}, a modern steel frame
track racing bicycle; and \emph{Yellow Bicycle}, a stripped down aluminum frame
road bicycle measured in two configurations, the second with the fork rotated
in the headtube 180 degrees for larger trail.

These eight different parameter sets can be used with, but are not
limited to, the benchmark bicycle model. The accuracy of all the measurements
are presented up through the eigenvalue prediction of the linear model. The
accuracies are based on error propagation theory with correlations taken into
account.

\begin{keywords}
bicycle,
parameters,
eigenvalues,
Bode.
\end{keywords}

\section{INTRODUCTION}
This work is intended to document the indirect measurement of eight real
bicycles' physical parameters. The physical parameters measured are those
needed for the benchmark Whipple bicycle model presented in
\cite{Meijaard2007}. The work is based on techniques used to measure the
instrumented bicycles in \cite{Kooijman2006}, \cite{Kooijman2008} and
\cite{Moore2009a}. We improve upon these methods by both increasing and reporting
the accuracies of the measurements and by measuring the complete moments of
inertia of the frame and fork needed for analysis of the nonlinear model.
Furthermore, very little data exists on the physical parameters of different
types of bicycles and this work aims to provide a small sample of bicycles.

D\"{o}hring~\cite{Dohring1953} and Singh and Goel~\cite{Singh1971} measured the
physical parameters of scooters.
Roland and Massing~\cite{Roland1971} measured the physical parameters of a
bicycle in much the
same way as is presented, including calculations of uncertainty from the indirect
measurement techniques. Patterson~\cite{Patterson2004} used a swing to measure the
inertia of a bicycle and rider. The present work is based on the work
done by Kooijman~\cite{Kooijman2006} using much of the same apparatus and
refining the measurement technique.
\section{BENCHMARK BICYCLE MODEL}
Recently, the Whipple bicycle model has been benchmarked~\cite{Meijaard2007}
and this model is widely used for bicycle dynamics studies. The unforced two
degree-of-freedom, $\mathbf{q}$ = [steer and roll], model takes the form:
\begin{equation}
    \mathbf{M\ddot{q}}
    +v\mathbf{C}_1\mathbf{\dot{q}}
    +\left[g\mathbf{K}_0
    +v^2\mathbf{K}_2\right]\mathbf{q}
    =0
    \label{eq:canonical}
\end{equation}
where the entries of the $\mathbf{M}$, $\mathbf{C}_1$, $\mathbf{K}_0$ and $\mathbf{K}_2$
matrices are combinations of 25 bicycle physical parameters that
include the geometry, mass, mass location and mass distribution of the four
rigid bodies. The 25 parameters presented in \cite{Meijaard2007} are not
necessarily a minimum set for the Whipple model, as shown in \cite{Sharp2008},
but are useful as they represent more intuitively measurable quantities.
Furthermore, many more parameters are not needed due to the assumptions of the
Whipple model such as no-slip tires, lateral symmetry, knife edge wheels, etc.
\section{BICYCLE DESCRIPTIONS}
We choose to measure the physical parameters of six bicycles
Fig.~\ref{fig:bicycles}. The three Batavus bicycles were donated by the
manufacturer. We asked for a bicycle that they considered stable and one that
they did not. They claimed the Browser was a ``stable'' bicycle and that the
Stratos was ``nervous''. The Fisher and the Pista were chosen to provide some
variety, a mountain and road bike. The yellow bike is used to demonstrate
bicycle stability.
\begin{description}
    \item[Batavus Browser (B, B*)]{The Batavus Browser Fig.~\ref{fig:browser}
        is an average priced Dutch city
        bike. It has a steel frame, a three speed internal rear hub, handle
        bars for an upright posture and includes various accessories for utility
        purposes. We measured the physical proprieties of the stock Browser model and
        also equipped with the instrumentation used in our other
        experiments~\cite{Kooijman2009}.}
    \item[Batavus Crescendo Deluxe (C)]{The Batavus Crescendo Deluxe
        Fig.~\ref{fig:crescendo} is also a
        Dutch city bike for touring. It has an aluminum frame, an
        eight speed internal hub, upright handlebars, accessories for utility
        and a suspension fork and suspension seatpost.}
    \item[Gary Fisher Ziggurat (G)]{The Gary Fisher Ziggurat Fig.~\ref{fig:fisher} is a modern
        lightweight front suspended mountain bike. It has a aluminum frame, large low
        pressure mountain bike tires, and is for racing with few extra accessories.}
    \item[Bianchi Pista (P)]{The Bianchi Pista Fig.~\ref{fig:pista} is a modern lightweight steel track
        bicycle. It has a single gear ratio and minimal extras to keep the
        weight low. It has drop handlebars and high pressure racing tires.}
    \item[Batavus Stratos Deluxe (S)]{The Batavus Stratos Deluxe
        Fig.~\ref{fig:stratos} is a sporty Dutch city bicycle. The frame is
        aluminum. It has a seven-speed internal hub and mountain
        style handle bars for a less upright seating posture, but also includes
        accessories for utility such as a rear rack, fenders, light and
        chainguard.}
    \item[Yellow Bicycle (Y, Y*)]{The yellow bicycle Fig.~\ref{fig:yellow} is used in the lab to
        demonstrate that a bicycle is stable at certain speeds. It is an
        aluminum road frame of unknown make with the most of components
        removed. The wheels, drop handlebar, seat, seat post and bottom bracket
        are the only parts on the bike. This bicycle was measured with both the
        fork in normal position and reversed. The fork was reversed to
        ''decrease the minimal stable speed``~\cite{Kooijman2006} of the bicycle.}
\end{description}
\begin{figure}[tb]
    \centering
    \subfloat[Batavus Browser]{
        \label{fig:browser}
        \includegraphics[width=1.75in]{../../../images/browserIns_sub.jpg}
        }
    \subfloat[Batavus Crescendo Deluxe]{
        \label{fig:crescendo}
        \includegraphics[width=1.75in]{../../../images/crescendo_sub.jpg}
        }
    \subfloat[Gary Fisher]{
        \label{fig:fisher}
        \includegraphics[width=1.75in]{../../../images/fisher_sub.jpg}
        }\\
    \subfloat[Bianchi Pista]{
        \label{fig:pista}
        \includegraphics[width=1.75in]{../../../images/pista_sub.jpg}
        }
    \subfloat[Batavus Stratos Deluxe]{
        \label{fig:stratos}
        \includegraphics[width=1.75in]{../../../images/stratos_sub.jpg}
        }
    \subfloat[Yellow Bicycle]{
        \label{fig:yellow}
        \includegraphics[width=1.75in]{../../../images/yellow_sub.jpg}
        }
    \caption{The six bicycles measured in the experiments. The Batavus
    Browser~\subref{fig:browser} is shown with the instrumentation and the
    Yellow Bicycle~\subref{fig:yellow} is shown with its fork reversed.}
    \label{fig:bicycles}
\end{figure}
\section{PARAMETERS}
The 25 parameters can be estimated using many techniques. Where possible we
measured the benchmark parameter directly.
\section{ACCURACY}
We took great care to improve and report the accuracy of the
measurements of the parameters. Following the thrust of
~\cite{Roland1971} we used error propagation theory to calculate accuracy of
the 25 benchmark parameters. We start by estimating the standard deviation of
the actual measurements taken. If $x$ is a parameter and is a function of
the measurements, $u,v,\ldots$, then $x$ is a random variable defined as
$x=f(u,v,\ldots)$. The sample variance of $x$ is defined as
\begin{equation}
    s_x^2 =
    \frac{1}{N-1}\sum^N_{i=1}
    \left[(u_i - \bar{u})^2\left(\frac{\partial x}{\partial u}\right)^2 +
    (v_i - \bar{v})^2\left(\frac{\partial x}{\partial v}\right)^2 +
    2(u_i - \bar{u})(v_i - \bar{v})\left(\frac{\partial x}{\partial u}\right)\left(\frac{\partial x}{\partial v}\right)
    + \ldots\right]
    \label{eqn:sampleVariance}
\end{equation}
Using the definitions for variance and covariance,
Equation~\ref{eqn:sampleVariance} can be simplified to
\begin{equation}
    s_x^2 = s_u^2\left(\frac{\partial x}{\partial u}\right)^2 +
            s_v^2\left(\frac{\partial x}{\partial v}\right)^2 +
            2s_{uv}\left(\frac{\partial x}{\partial u}\right)\left(\frac{\partial x}{\partial v}\right)
            + \ldots
    \label{eqn:variance}
\end{equation}
If $u$ and $v$ are uncorrelated then $s_{uv}=0$. Most of the calculations
hereafter have uncorrelated variables but a few do not and the covariance has to
be taken into account. Equation~\ref{eqn:variance} can be
used to calculated the variance of all types of functions. Simple addition of
two random variables may be the most basic example:
\begin{eqnarray}
    \label{eqn:addition}
    x &=&  au + bv\\
    s_x &=& a^2s_u^2 + b^2s_v^2
\end{eqnarray}
\section{GEOMETRY}
\subsection{WHEEL RADII}
The radii of the front $r_\mathrm{F}$ and rear $r_\mathrm{R}$ wheels were
estimated by measuring the linear distance traversed along the ground through
either 13 or 14 rotations of the wheel. Each wheel was measured separately and
the measurements were taken with a 72kg rider seated on the bicycle. A 30 meter
tape measure (resolution: 2mm) was pulled tight and taped on a flat level smooth floor. The tire
was marked with chalk and aligned with the tape measure
Fig.~\ref{fig:tireChalk}. The accuracy of the
distance measurement is approximately $\pm0.01$m. The tires were pumped to the
recommended inflation pressure before the measurements. The wheel radius is
calculated by
\begin{equation}
	r\pm\sigma_r=
    \frac{d}{2\pi n}
    \pm\left(\frac{\sigma_d}{2\pi n}\right)
	\label{eq:wheelRadius}
\end{equation}
\begin{figure}[tb]
	\begin{center}
		\includegraphics[width=4in]{../../../images/tireChalk.jpg}
	\end{center}
	\caption{Wheel and tire with chalk mark aligned to the tape measure.}
	\label{fig:tireChalk}
\end{figure}
\subsection{HEAD TUBE ANGLE}
\label{sec:headtube}
The head tube angle was measured directly using an electronic level with a
$\pm0.2^{\circ}$ accuracy. The bicycle frame was fixed perpendicular to the
ground, the steering angle was set to the nominal, tire pressures were at
recommended levels and the bicycle was unloaded. The steer axis
tilt $\lambda$ is the complement to the head tube angle.
\begin{equation}
    \lambda\pm\sigma_\lambda
    =\frac{\pi}{180^{\circ}}(90^{\circ}-\lambda_{ht})
    \pm\left(\frac{\pi}{180^\circ}\right)\sigma_{\lambda_{ht}}
\label{eq:headTubeAngle}
\end{equation}
\subsection{TRAIL}
Trail is difficult to measure directly so we instead chose to measure the fork
offset. The fork offset was measured by clamping the steer tube of the front
fork into a v-block on a flat table. A ruler was used to measure the height of
the center of the head tube and the height of the center of the axle axis. The
fork blades were aligned such that the axle axis was parallel to the table
surface.
\begin{equation}
	c=\frac{r_\mathrm{F}\sin{\lambda}-f_o}{\cos{\lambda}}
	\label{eq:trail}
\end{equation}
\begin{equation}
    \sigma_{c}^{2}=\sigma_{r_{\mathrm{F}}}^{2}\tan^2{\lambda} -
    \sigma_{f_o}^{2}\sec^2{\lambda} +
    \sigma_{\lambda}^{2}\left(r_\mathrm{F}\sec^2{\lambda} -
    f_o\sec{\lambda}\tan{\lambda}\right)^2
    \label{eq:TrailVar}
\end{equation}
\subsection{WHEELBASE}
We measured the wheelbase with the bicycle in nominal
configuration described in Sec.~\ref{sec:headtube}. We used a tape measure
to measure the distance from one wheel axle center to the other with a 0.002 m
accuracy.
\section{MASS}
\begin{figure}[tb]
    \begin{center}
        \includegraphics[width=3in]{../../../images/massScale.jpg}
    \end{center}
    \caption{The scale used to measure the mass of each bicycle component.}
    \label{fig:massScale}
\end{figure}
The total mass of each bicycle was measured using a spring scale with a
resolution of 100 grams. The total mass was only used for comparison purposes.
Each of the four bicycle parts were measured using a Molen 20 kilogram scale with a
resolution of 20 grams. The accuracy was conservatively assumed to also be
$\pm20$ grams.
\section{CENTER OF MASS}
\subsection{WHEELS}
The centers of mass of the wheels are assumed to be at their geometrical
centers to comply with the Whipple model.
\subsection{REAR FRAME}
The rear frame was hung in three orientations as a torsional pendulum (both for
the center of mass measurements and the moment of inertia measurements
described in Sec.~\ref{sec:moi}). We assumed that the frame was laterally
symmetric, complying with the Whipple model. The frame could rotate about a joint
such that gravity aligned the center of mass with the pendulum axis. The
orientation angle of the headtube, $\alpha_\mathrm{B}$, Fig.~\ref{fig:angles} relative
to the earth was measured using a digital level ($\pm0.2^{\circ}$ accuracy),
Figure~\ref{fig:level}. A string was aligned with the pendulum axis
and allowed to pass by the frame. The horizontal distance $a_\mathrm{B}$ between the rear
axle and the string was measured by aligning a ruler perpendicular to
the string. The distance $a_\mathrm{B}$ was negative if the string fell to the right of
the rear axle and positive if it fell to the left of the rear axle. These
measurements allow for the calculation of the center of mass location in the
global reference frame.
\begin{figure}[tb]
    \centering
    \subfloat[]{\label{fig:angles}\includegraphics[width=3in]{../../../figures/angles.pdf}}
    \subfloat[]{\label{fig:triangle}\includegraphics[width=2.75in]{../../../figures/triangle.pdf}}
    \caption{\subref{fig:angles} Pictorial description of the angles and dimensions that related
    the nominal bicycle reference frame $XYZ_B$ with the pendulum reference frame
    $XYZ_P$. \subref{fig:triangle} Exaggerated intersection of the three pendulum axes and the
    location of the center of mass.}
\end{figure}
\begin{figure}[tb]
    \centering
    \subfloat[]{\label{fig:level}\includegraphics[width=2.75in]{../../../images/YellowFrameTorsionalThird.jpg}}
    \quad
    \subfloat[]{\label{fig:PendDist}\includegraphics[width=2.75in]{../../../images/pendDist.jpg}}
    \caption{\subref{fig:level} The digital level was mounted to a straight edge aligned
    with the headtube of the bicycle frame. This was done without allowing the
    straight edge to touch the frame. The frame wasn't completely stationary so
    this was difficult. The light frame oscillations could be damped out by
    submerging a low hanging area of the frame into a bucket of water to
    decrease the oscillation. \subref{fig:PendDist} Measuring the distance from
    the pendulum axis to the rear wheel axle using level ruler.}
\end{figure}
The frame rotation angle $\beta_\mathrm{B}$ is defined as rotation of the frame in the
nominal configuration to the hanging orientation, rotated about the $Y$ axis.
\begin{equation}
    \beta=\lambda-\alpha
    \label{eq:frameRotAng}
\end{equation}
\begin{equation}
    \sigma_{\beta}^{2} = \sigma_{\lambda}^{2} + \sigma_{\alpha}^{2}
    \label{eq:FrameRotAngVar}
\end{equation}
The center of mass can be found by realizing that the pendulum axis $X_P$ is
simply a line in the nominal bicycle reference frame with a slope $m$ and a
z-intercept $b$ where the $i$ subscript corresponds the different frame
orientations Fig.~\ref{fig:triangle}. The slope can be shown to be
\begin{equation}
	m_i=-\tan{\beta_i}
\label{eq:slope}
\end{equation}
\begin{equation}
    \sigma_{m}^{2} = \sigma_{\beta}^{2}\sec^{4}{\beta}
    \label{eq:SlopeVar}
\end{equation}
The z-intercept can be shown to be
\begin{equation}
    b_i=-\left(\frac{a_\mathrm{B}}{\cos{\beta_i}}+r_\mathrm{R}\right)
    \label{eq:zInt}
\end{equation}
\begin{equation}
    \sigma_{b}^{2} = \sigma_{a}^{2}\sec^{2}{\beta} +
    \sigma_{r_\mathrm{R}}^{2} +
    \sigma_{\beta}^{2}a^{2}\sec^{2}{\beta}\tan^{2}{\beta}
    \label{eq:zIntvar}
\end{equation}
Theoretically, the center of mass lies on each line but due to experimental
error, if there are more than two lines, the lines do not cross all at the same point. Only two lines
are required to calculate the center of mass of the laterally
symmetric frame, but more orientations increase the center of mass measurement
accuracy. The three lines are defined as:
\begin{equation}
   z = m_ix+b_i
   \label{eq:line}
\end{equation}
The mass center location can be calculated by finding the intersection of these three
lines. Two approaches were used used to calculate the center of mass. Intuition
lead us to think that the center of mass is located at the centroid of the
triangle made by the three intersecting lines. The centroid can be found by
calculating the intersection point of each pair of lines and then averaging the
three intersection points.
\begin{equation}
	\left[
	\begin{array}{cc}
		-m_1 & 1\\
		-m_2 & 1
	\end{array}
	\right]
	\left[
	\begin{array}{c}
		x_a\\
		z_a
	\end{array}
	\right]
	=
	\left[
	\begin{array}{c}
		b_1\\
		b_2
	\end{array}
	\right]
\label{eq:linearSystem}
\end{equation}
\begin{equation}
    x_\mathrm{B} = \frac{x_a + x_b + x_c}{3}
\end{equation}
\begin{equation}
    z_\mathrm{B} = \frac{z_a + z_b + z_c}{3}
\end{equation}
Alternatively, the three lines can be treated as an over determined linear
system and the least squares method is used to find a unique solution. This
solution is not the same as the triangle centroid method.
\begin{equation}
	\left[
	\begin{array}{cc}
		-m_1 & 1\\
		-m_2 & 1\\
		-m_3 & 1
	\end{array}
	\right]
	\left[
	\begin{array}{c}
        x_\mathrm{B}\\
        z_\mathrm{B}
	\end{array}
	\right]
	=
	\left[
	\begin{array}{c}
		b_1\\
		b_2\\
		b_3
	\end{array}
	\right]	
\label{eq:leastSquares}
\end{equation}
The solution with the higher accuracy is the preferred one.
\subsection{Fork}
The fork and handlebars are a bit trickier to hang in three different
orientations. Typically two angles can be obtained by clamping to the steer
tube at the top and the bottom. The third angle can be obtained by clamping to
the stem. The center of mass of the fork is calculated in the same fashion. The
slope of the line in the benchmark reference frame is the same as for the
frame but the z-intercept is different:
\begin{equation}
    b = w\tan{\beta} - r_\mathrm{F} - \frac{a}{\cos{\beta}} 
    \label{eq:zIntFork}
\end{equation}
\begin{equation}
    \sigma_{b}^{2} = \sigma_{w}^{2}\tan^{2}\beta +
    \sigma_{\beta}^{2}\left(w\sec^{2}\beta -
    a\sec\beta\tan\beta\right)^{2} + \sigma_{r_\mathrm{F}}^{2} +
    \sigma_{a}^{2}\sec^{2}\beta
    \label{eq:zIntForkVar}
\end{equation}
\section{MOMENT OF INERTIA}
\label{sec:moi}
The moments of inertia of the wheels, frame and fork were measured by taking
advantage of the assumed symmetry of the parts and by hanging the parts as both
compound and torsional pendulums and measuring their periods of oscillation when
perturbed at small angles. The rate of oscillation was measured
using a \href{http://www.siliconsensing.com/CRS03}{Silicon Sensing CRS03 100
deg/s rate gyro}. The rate gyro was sampled at
1000hz with a
\href{http://sine.ni.com/nips/cds/view/p/lang/en/nid/14604}{National
Instruments USB-6008 12 bit data acquisition unit} and
\textsc{Matlab}. The measurement durations were either 15 or 30 secs and each moment of
inertia measurement was performed three times. No extra care was taken to
calibrate the rate gyro, maintain a constant power source (i.e. the battery
drains slowly), or account for drift. The raw voltage signal was used to
determine only the period of oscillation which is needed for the moment of inertia
calculations.
\begin{figure}[tb]
    \begin{center}
        \includegraphics[width=4in]{../../../plots/PendFit/BrowserFrameCompoundFirst1.png}
    \end{center}
    \caption{Example of the raw voltage data taken during a 30 second
    measurement of the oscillation of one of the components.}
    \label{fig:voltage}
\end{figure}
The function Eqn~\ref{eqn:decayOs} was fit to the data using a nonlinear least squares fit
routine for each experiment to determine the quantities $A$, $B$, $C$, $\zeta$,
and $\omega$.
\begin{equation}
    f(t) = A + e^{-\zeta\omega t}\left[B\sin{\sqrt{1-\zeta^2}\omega t} +
    C\cos{\sqrt{1-\zeta^2}\omega t}\right]
    \label{eqn:decayOs}
\end{equation}
Most of the data fit the damped oscillation function well with very light (and
ignorable) damping. There were several instances of beating-like phenomena for
some of the parts at particular orientations. Roland and
Massing~\cite{Roland1971} also encountered this problem and used a bearing to
prevent the torsional pendulum from swinging. Figure~\ref{fig:beating} shows an
example of the beating like phenomena.
\begin{figure}[tpb]
    \begin{center}
        \includegraphics[width=4in]{../../../plots/PendFit/CrescendoForkTorsionalFirst2.png}
    \end{center}
    \caption{An example of the beating-like phenomena observed on 5\% of the
    experiments.}
    \label{fig:beating}
\end{figure}

The physical phenomenon observed corresponding to data sets such as these was that
the bicycle frame or fork was perturbed torsionally. After set into motion the
torsional motion died out and a longitudinal swinging motion increased. The
motions alternated back and forth with neither ever reaching zero. The
frequencies of these motions were very close to one another and it is not
apparent how dissect the two. We explored fitting to a function such as
\begin{equation}
    f(t) = A\sin{(\omega_1 t)} + B\sin{(\omega_2 t + \phi)} + C
    \label{eqn:sumSines}
\end{equation}
But the fit predicts that $\omega_1$ and $\omega_2$ are very similar
frequencies. There was no easy way to choose which of the two $\omega$'s was
the one associated with the torsional oscillation. Some work was done to model
the torsional pendulum as a laterally flexible beam to determine this, but we
thought accuracy of the period calculation would not
improve enough for the effort required. Future experiments should simply
prevent the swinging motion of the pendulum without damping the torsional
motion.

The period for a damped oscillation is
\begin{equation}
    T = \frac{2\pi}{\sqrt{1-\zeta^2}\omega_n}
    \label{eqn:periodDamped}
\end{equation}
The uncertainty in the period, $T$, can be determined from the fit. Firstly,
the variance of the fit is
\begin{equation}
    \sigma_y^2 =
    \frac{1}{N-5}\sum_{i=1}^N(y_{mi}-\bar{y}_m)^2-(y_{pi}-\bar{y}_m)^2
    \label{eqn:fitVariance}
\end{equation}
The covariance matrix of the fit function can be formed
\begin{equation}
    \mathbf{U} = \sigma_y^2\mathbf{H}^{-1}
    \label{eqn:covariance}
\end{equation}
where $\mathbf{H}$ is the Hessian~\cite{Hubbard1989b}. $\mathbf{U}$ is a $5\times5$ matrix with the
variances of each of the five fit parameters along the diagonal.
The variance of $T$ can be computed using the variance of $\zeta$ and $\omega$. It
is important to note that the uncertainties in the period are very low
($<1e-4$), even for the fits with low $r^2$ values.
\subsection{TORSIONAL PENDULUM}
A torsional pendulum was used to measure all moments of inertia about axes
in the laterally symmetric plane of each of the wheels, fork and frame. The
pendulum is made up of a rigid mount, an upper clamp, a torsion rod, and
various lower clamps.
\begin{figure}[tb]
    \begin{center}
        \includegraphics[width=2in]{../../../images/fixture.jpg}
    \end{center}
    \caption{The rigid pendulum fixture mounted to a concrete column.}
    \label{fig:fixture}
\end{figure}
A 5 mm diameter, 1 m long mild steel rod was used as the torsion
spring. A lightweight, low relative moment of inertia clamp was constructed
that could clamp the rim and the tire. The moments of inertia of the clamps
were neglected. The wheel was hung freely such that the
center of mass aligned with the torsional pendulum axis and then
secured. The wheel was then perturbed and oscillated about the pendulum
axis. The rate gyro was mounted on the clamp oriented along the pendulum
axis.

The torsional pendulum was calibrated using a known moment of
inertia Fig.~\ref{fig:rod}. A torsional pendulum almost identical to the one used in
\cite{Kooijman2006} was used to measure the average period $\overline{T}_i$ of
oscillation of the rear frame at three different
orientation angles $\beta_i$, where $i=1$, $2$, $3$, as shown in
Fig.~\ref{fig:triangle}. The parts were perturbed lightly, less than 1 degree,
and allowed to oscillate about the pendulum axis through at least ten periods.
This was done at least three times for each frame and the recorded periods were
averaged.
\begin{figure}[tb]
    \begin{center}
        \includegraphics[width=4in]{../../../images/rod.jpg}
    \end{center}
    \caption{The steel calibration rod. The moment of inertia of the rod,
    $I=\frac{m}{12}(3r^2+l^2)$, can be used to estimate the stiffness of the
    pendulum, $k=\frac{4I\pi^2}{\overline{T}^2}$, with $k=5.62\pm0.02$ $\frac{\textrm{Nm}}{\textrm{rad}}$}
    \label{fig:rod}
\end{figure}
\subsection{WHEELS}
Finding the full inertia tensors of the wheels is less complex because the wheels
are assumed symmetric about three orthogonal planes so products of
inertia are zero. The $I_{xx}=I_{zz}$ moments of inertia were calculated by measuring
the averaged period of oscillation about an axis in the $XZ$-plane using the
torsional pendulum setup and Eq.~\ref{eq:torPend}.The wheels are assumed to be
laterally symmetric and about any radial axis. Thus only two
moments of inertia are required for the set of benchmark parameters. The moment
of inertia about the axle was measured by hanging the wheel as a compound
pendulum, Fig.~\ref{fig:wheelIyy}. The wheel was hung on a
horizontal rod and perturbed to oscillate about the axis of the rod. This rate
gyro was attached to the spokes near the hub and oriented mostly along the
axle axis. The wheels tended to precess at the contact point about the vertical
axis which added a very low frequency component of rate along the vertical
radial axis, but this should not affect the period estimation about the compound
pendulum axis. A fixture that prevented precession would be preferable for
future measurements. The pendulum arm length is the distance
from the rod/rim contact point to the mass center of the wheel. The inner
diameter of the rim was measured and divided by two to get $l_\mathrm{F,R}$. The moment of
inertia about the axle is calculated from:
\begin{equation}
    I_{\mathrm{R}yy} = \left(\frac{\bar{T}}{2\pi}\right)^2m_\mathrm{R}gl_\mathrm{R} -
    m_\mathrm{R}l^2
    \label{eq:CompoundInertia}
\end{equation}
%\begin{equation}
    %need the sigma for IRyy
%\end{equation}
\begin{figure}[tb]
    \centering
    \subfloat[]{
        \label{fig:FwheelTor}
        \includegraphics[width=2.75in]{../../../images/CrescendoFwheelTorsionalFirst.jpg}
    }
    \quad
    \subfloat[]{
        \label{fig:wheelIyy}
        \includegraphics[width=2.75in]{../../../images/wheelIyy.jpg}
    }
    \caption{\subref{fig:FwheelTor} The front wheel of the Crescendo hung as a
    torsional pendulum. \subref{fig:wheelIyy} A wheel hung as a compound pendulum.}
    \label{fig:wheelPend}
\end{figure}
The radial moment of inertia was measured by hanging the wheel as a torsional
pendulum, Fig.~\ref{fig:FwheelTor}. The wheel was hung freely such that the center of mass aligned with
the torsional pendulum axis and then secured. The wheel was then
perturbed and oscillated about the vertical pendulum axis. The radial moment of inertia can can calculated as such:
\begin{equation}
    I_{xx} = \frac{k\bar{T}^2}{4\pi^2}
\end{equation}
%\begin{equation}
    %need the sigma for Ixx
%\end{equation}
\subsection{FRAME}
Three measurements were made to estimate the globally referenced moments and
products of inertia ($I_{xx}$, $I_{xz}$ and $I_{zz}$) of the rear frame. The
frame was typically hung from the three main tubes: seat tube, down tube
and top tube, Fig.~\ref{fig:level}. The rear fender prevented easy connection to the seat tube on
some of the bikes and the clamp was attached to the fender. The fender was
generally less rigid than the frame tube. For best accuracy with only three
orientation angles, the frame should be hung at three angles that are
$120^\circ$ apart. The three tubes on the frame generally provide that the
orientation angles were spread evenly at about $120^\circ$. Furthermore, taking
data at more orientation angles could improve the accuracy and
is generally possible with standard diamond frame bicycles.

Three moments of inertia $J_{i}$ about the pendulum axes were calculated using
Eq.~\ref{eq:torPend}.
\begin{equation}
	J_i=\frac{k\overline{T}_i^2}{4\pi^2}
\label{eq:torPend}
\end{equation}
%\begin{equation}
    %need sigma
%\end{equation}
The moments and products of inertia of the rear frame and handlebar/fork
assembly with reference to the benchmark coordinate system were calculated by
formulating the relationship between inertial frames
\begin{equation}
	\mathbf{J}_i=\mathbf{R}_i\mathbf{IR}_i^T
\label{eq:rotIn}
\end{equation}
where $\mathbf{J}_i$ is the inertia tensor about the pendulum axes,
$\mathbf{I}$, is the inertia tensor in the global reference frame and
$\mathbf{R}$ is the rotation matrix relating the two frames,
Fig.~\ref{fig:angles}. The global inertia
tensor is defined as
\begin{equation}
	\mathbf{I}=
	\left[
	\begin{array}{rr}
		I_{xx}  & I_{xz}\\
		I_{xz} & I_{zz}
	\end{array}
	\right]\textrm{.}
	\label{eq:MoI}
\end{equation}
The inertia tensor can be reduced to a $2\times2$ matrix because the frame is
assumed to be laterally symmetric and the $y$ axis of the pendulum reference is
the same as the $y$ axis of the benchmark reference frame. The simple rotation
matrix about the $Y$-axis
can similarly be reduced to a $2\times2$ matrix where $s_{\beta i}$ and
$c_{\beta i}$ are defined as $\sin{\beta_i}$ and $\cos{\beta_i}$,
respectively.
\begin{equation}
	\mathbf{R}=
	\left[
	\begin{array}{rr}
		c_{\beta i} & -s_{\beta i}\\
		s_{\beta i} & c_{\beta i}
	\end{array}
	\right]
	\label{eq:rotMat}
\end{equation}
The first entry of $\mathbf{J}_i$ in Eq.~\ref{eq:rotIn} is the moment of
inertia about the pendulum axis and is written explicitly as
\begin{equation}
	J_{i}=c^{2}_{\beta i}I_{xx}-2s_{\beta i}c_{\beta i}I_{xz}+s^{2}_{\beta i}I_{zz}\textrm{.}
\label{eq:inRelComp}
\end{equation}
Similarly, calculating all three $J_{i}$ allows one to form
\begin{equation}
	\left[
	\begin{array}{c}
		J_{1}\\
		J_{2}\\
		J_{3}
	\end{array}
	\right]
	=
	\left[
	\begin{array}{ccc}
		c_{\beta 1}^2 & -2s_{\beta 1}c_{\beta 1} & s_{\beta 1}^2\\
		c_{\beta 2}^2 & -2s_{\beta 2}c_{\beta 2} & s_{\beta 2}^2\\
		c_{\beta 3}^2 & -2s_{\beta 3}c_{\beta 3} & s_{\beta 3}^2
	\end{array}
	\right]
	\left[
	\begin{array}{c}
		I_{xx}\\
		I_{xz}\\
		I_{zz}
	\end{array}
	\right]
\label{eq:inRel}
\end{equation}
and the moments of inertia can be solved for.
The inertia of the frame about an axis normal to the plane of symmetry was
estimated by hanging the frame as a compound pendulum at the wheel axis,
Fig.~\ref{fig:frameCompound}. Equation~\ref{eq:CompoundInertia} is used but
with the mass of the frame and the frame pendulum length.
\begin{equation}
    l_B=\sqrt{x_B^2+(z_B+r_R)^2}
    \label{eq:FramePendLength}
\end{equation}
\begin{figure}[tb]
    \centering
    \subfloat[]{
        \label{fig:frameCompound}
        \includegraphics[width=2in]{../../../images/YellowFrameCompoundFirst.jpg}
        }
    \subfloat[]{
        \label{fig:forkCompound}
        \includegraphics[width=2in]{../../../images/BrowserInsForkCompoundFirst.jpg}
        }
        \caption{\subref{fig:frameCompound} Rear frame hung as a compound
        pendulum. \subref{fig:forkCompound} Browser fork hung as a
        compound pendulum.}
        \label{fig:compound}
\end{figure}
\subsection{FORK AND HANDLEBAR}
The inertia of the fork and handlebar is calculated in the same way as the frame. The fork is
hung as both a torsional pendulum, Fig.~\ref{fig:StratosFork}, and as a compound pendulum,
Fig.~\ref{fig:forkCompound}. The fork provides fewer mounting options to
obtain at least three equally spaced orientation angles, especially if there is
no fender. The torsional calculations follow equations~\ref{eq:torPend} through
\ref{eq:inRel} and the compound pendulum calculations is calculated with
equation~\ref{eq:CompoundInertia}. The fork pendulum length is calculated using
\begin{equation}
    l_H=\sqrt{(x_H-w)^2+(z_B+rF)^2}
\end{equation}
\begin{figure}[tb]
    \centering
    \includegraphics[width=2.75in]{../../../images/StratosForkTorsionalThird.jpg}
    \caption{The Stratos fork and handlebar assembly hung as a torsional
        pendulum.}
    \label{fig:StratosFork}
\end{figure}
\section{LINEAR ANALYSIS}
Once all bicycle parameters have been calculated the canonical matrices
can be formed and the linear dynamics of the bicycles can be explored. The
values of the canonical matrices can be found in the second table in Appendix~\ref{sec:partables}. We also
added the same rigid rider to each bicycle for further comparison. The rigid
rider was assumed to be in the same position and posture for each bicycle relative
to the rear wheel contact point.
\begin{table}[tb]
    \centering
    \caption{Mass, center of mass and moment of inertia for the rider relative
    to the benchmark coordinate system from~\cite{Moore2009a}}
    \begin{tabular}{ll}
        Parameter & Value\\
        \hline
        $m_\mathrm{P}$ [kg] & 72\\
        $x_\mathrm{P}$ [m] & 0.2909\\
        $z_\mathrm{P}$ [m] & -1.1091\\
        $I_\mathrm{P}$ [$\mathrm{kg\ m}^2$] &
            $\left[
            \begin{array}{ccc}
                 7.9985 & 0      & -1.9272\\
                 0      & 8.0689 & 0\\
                -1.9272 & 0      & 2.3624
            \end{array}
            \right]$
    \end{tabular}
    \label{tab:riderParam}
\end{table}
\subsection{EIGENVALUES}
\begin{figure}[tbp]
    \centering
    \includegraphics[width=6in]{../../../plots/Bike/eig_plot.pdf}
    \caption{Eigenvalues versus speed for all eight bicycles without the rider.}
    \label{fig:bikeEigPlot}
\end{figure}
The eigenvalues of the bicycles with (Fig.~\ref{fig:bikeEigPlot}) and without
(Fig.~\ref{fig:bikeRiderEigPlot} the rider can be plotted versus forward speed.
Figure~\ref{fig:bikeEigPlot} shows that the bikes have the typical
characteristics of the benchmark bicycle: four real roots at very slow
speeds, two of which are unstable; a complex pair that is unstable at lower
speeds and stable at intermediate speeds; and a root that is mildly unstable at
higher speeds. The one noticeable difference is that the capsize and caster
modes are contained in a complex pair between about 0.5 and 3 m/s. The frequency
of oscillation is of comparable magnitude to that of the weave mode. But, the root locus in the real
and imaginary plane, Fig.~\ref{fig:rootloci}, shows that the mode damps out
quickly. Examining the eigenvectors reveals that the mode is steer leading
roll with a 90 degree phase, both of their magnitudes being similar,
Fig~\ref{fig:evec}.
\begin{figure}[tb]
    \begin{center}
        \includegraphics[width=6in]{../../../plots/Bike/CrescendoRootLoci.pdf}
    \end{center}
    \caption{The root loci with speed as the parameter for the Crescendo.}
    \label{fig:rootloci}
\end{figure}
\begin{figure}[tb]
    \begin{center}
        \includegraphics[width=3in]{../../../plots/cres.pdf}
    \end{center}
    \caption{Eigenvector components for the second complex mode
    pair at low speed.}
    \label{fig:evec}
\end{figure}
With the rider added, the second complex pair disappears and the bikes have the
typical characteristics of the benchmark bicycle model. Reversing the fork on the yellow
bike lowers the weave critical speed and increases the stable speed range.
Also, the addition of weight to the rear rack of the Browser does little to the
eigenvalues.
\begin{figure}[tbp]
    \centering
    \includegraphics[width=6in]{../../../plots/BikeRider/eig_plot.pdf}
    \caption{Eigenvalues versus speed for all eight bicycles with the same
    rigid rider.}
    \label{fig:bikeRiderEigPlot}
\end{figure}
\subsection{FREQUENCY RESPONSE}
The frequency response of the bicycles (Fig.~\ref{fig:bikeBode}) and bicycle
with rider (Fig.~\ref{fig:bikeRiderBode}) also reveal some
interesting things. In the steer-torque-to-roll Bode diagram the magnitude
difference among bicycles can vary up to 10 dB (or about 8.5 degrees per Newton-meter of
torque) for the particular speed shown. The difference in the frequency response for the bicycle with the
rigid rider shows less variation among the bicycles,
Fig.~\ref{fig:bikeRiderBode}, as the rider's mass and inertia play a larger roll.
\begin{figure}[tbp]
    \centering
    \includegraphics[width=6in]{../../../plots/Bike/Bode/Tdel2phi.pdf}
    \caption{The frequency response for steer-torque-to-roll for all eight
    bicycles without the rider at 2 m/s.}
    \label{fig:bikeBode}
\end{figure}
\begin{figure}[tbp]
    \centering
    \includegraphics[width=6in]{../../../plots/BikeRider/Bode/Tdel2phi.pdf}
    \caption{The frequency response for steer-torque-to-roll for all eight
    bicycles with the same rigid rider at 2 m/s.}
    \label{fig:bikeRiderBode}
\end{figure}
\section{CONCLUSION}
We have presented a detailed method to accurately estimate the physical parameters
of a bicycle needed for the benchmarked Whipple bicycle
model~\cite{Meijaard2007}. We measured eight different bicycles providing
both the parameter sets and linear model coefficient matrices
for the bicycles alone and the bicycles with the same
rigid rider. The uncertainties in the parameters and matrix coefficients are
included for the bicycle alone. Finally, we have presented a brief comparison of the
eight bicycles using eigenanalysis and Bode frequency response.
\section{ACKNOWLEDGEMENTS}
This material is based upon work partially supported by the National Science
Foundation under Grant No. 0928339.

\bibliographystyle{acm}
\bibliography{bicycle}
\appendix
\section{PARAMETER TABLES}
\label{sec:partables}
The tabulated values for the both the physical parameters and the canonical matrix
coefficients are shown in the following four tables. The uncertainties in the
estimations of both the parameters and coefficients are also shown for the
bicycle without a rider.
%\section{RAW DATA}
%\begin{landscape}
%\subfile{../../../tables/Bike/Parameters/MasterParTable.tex}
%\end{landscape}
%\section{PARAMETERS}
%\subsection{BICYCLE}
\begin{landscape}

\begin{table}[tb]
\caption{The parameters for the eight bicycles with uncertainties in the
estimations.}
\subfile{../../../tables/Bike/Parameters/MasterParTable.tex}
\label{tab:bicyclePar}
\end{table}

\begin{table}[tb]
\caption{The canonical matrix coefficients for the eight bicycles with the
uncertainty in the estimations.}
\subfile{../../../tables/Bike/Canonical/MasterCanTable.tex}
\label{tab:bicycleCan}
\end{table}

\begin{table}[tb]
\caption{The parameters for the eight bicycles with the same rigid rider.}
\subfile{../../../tables/BikeRider/Parameters/MasterParTable.tex}
\label{tab:bicycleRiderPar}
\end{table}

\begin{table}[tb]
\caption{The canonical matrix coefficients for the eight bicycles with the
rigid rider.}
\subfile{../../../tables/BikeRider/Canonical/MasterCanTable.tex}
\label{tab:bicycleRiderCan}
\end{table}

\end{landscape}
\end{document}
