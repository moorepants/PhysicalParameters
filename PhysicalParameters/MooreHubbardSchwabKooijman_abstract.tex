\documentclass{bmd2010a}

\begin{document}

\begin{flushleft}
{\fontsize{16pt}{20pt}\selectfont%
  An Accurate Method of Measuring and Comparing a Bicycle's Physical Parameters\\}
\end{flushleft}

%%%%%%%%%%%%%%%% authors %%%%%%%%%%%%%%%
\begin{flushleft}
  {\fontsize{12}{14}{Jason K. Moore, Mont Hubbard and Dale L. Peterson}\\}
  \textit{Mechanical and Aerospace Engineering\\
          University of California, Davis\\
          One Shields Avenue, Davis, CA 95616, USA\\
          e-mail: jkmoor@ucdavis.edu, mhubbard@ucdavis.edu,
          dlpeterson@ucdavis.edu
  }\vspace{10pt}\\
  {\fontsize{12}{14}{A. L. Schwab and J. D. G. Kooijman}\\}
  \textit{Laboratory for Engineering Mechanics\\
          Delft University of Technology, Mekelweg 2, 2628CD Delft, The
          Netherlands\\
          e-mail: a.l.schwab@tudelft.nl, jodikooijman@gmail.com
  }\vspace{10pt}\\
\end{flushleft}

\section*{Abstract}
Accurate measurements of a bicycle's physical parameters are required for
realistic dynamic simulations. For the most basic models the geometry, mass,
mass center location and mass distributions must be measured. More complex models
require measurements of tire characteristics, stiffness, damping, etc. This
paper concerns the measurement of bicycle parameters required for
simulations of the benchmark bicycle presented in \cite{Meijaard2007}. This
model is composed of four rigid bodies, has ideal rolling and frictionless joints,
and is assumed to be laterally symmetric. A set of 25
physical parameters is used to describe the geometry, mass, mass location and
mass distribution of each of the rigid bodies. These parameters are presented
relative to a reference frame when the bicycle is in the nominal upright
configuration. The methods described herein are based primarily on the work
done in \cite{Kooijman2006} but have been refined for improved accuracy and
methodology. Previously, \cite{Roland1971} measured an experimental bicycle frame in a
similar fashion and both \cite{Dohring1953} and \cite{Singh1971} have used
similar techniques with scooters. We measured the characteristics of six
different bicycles, two of which were set up in two different configurations.
This is a total of eight different parameter sets that can be used with, but not
limited to, the benchmark bicycle model. The parameters set is also complete
for non-linear benchmark bicycle model.

The six bicycles, chosen for both variety and convenience, are as follows:
\begin{description}
    \item[Batavus Browser] The Dutch style city bicycle was measured
        in two configurations. One with the instrumentation described in
        \cite{Kooijman2009a} and one as received from the manufacturer.
    \item[Batavus Stratos Deluxe] A Dutch style sporty city bicycle.
    \item[Batavus Crescendo Deluxe] A Dutch style city bicycle with a
        suspension fork.
    \item[Gary Fisher Mountain Bike] A hardtail mountain bicycle.
    \item[Bianchi Pista] A modern steel frame track racing bicycle.
    \item[Yellow Bicycle] A stripped down aluminum frame road bicycle. This
        bike was measured in two configurations, the second with the fork
        rotated in the headtube 180 degrees for larger trail.
\end{description}
\subsection*{Geometry}
The benchmark bicycle requires five basic geometric measurements: wheelbase
$w$, steer axis tilt $\lambda$, trail $c$, and wheel radii $r_\mathrm{F}$
and $r_\mathrm{R}$. The wheelbase and steer axis tilt were measured directly
with a meter scale and a digital level, respectively. The trail and wheel radii were
measured indirectly using fork offset and loaded wheel circumference.
\subsection*{Mass, Mass Center Location and Mass Distribution}
The mass of each of the rigid bodies were measured with a precision scale.
The centers of mass of each of the rigid bodies were measured by hanging each
rigid body at the mass center and assuming lateral symmetry. The mass centers
were calculated from the orientation angle of the headtube and the location of
the point of suspension relative to the wheel center.
The wheel centers of mass were assumed to be in their ideal
geometric center.

The moments of inertia of the rigid bodies were measured in both torsional
pendulum and compound pendulum configurations. Again the assumed symmetry of the
bicycle was utilized to reduce the number of measurements needed. The
in-symmetric-plane moments of inertia were calculated from measurements of the
rigid bodies' periods of oscillation when hung as a torsional pendulum at
different orientation angles. The out-of-plane moments of inertia were
measured by swinging the rigid bodies as a compound pendulum.
\subsection*{Results}
Parameter sets for all six bicycles and all eight
configurations, example in Table~\ref{tab:par},  will be
presented, together with basic comparisons of the open loop dynamics of the
different types of bicycles. The physical parameters, the mass, damping and stiffness
matrices of the system, the eigenvalues and open loop transfer functions at
different speeds will be compared with typical anecdotal
descriptions of handling.
\begin{table*}[tb]
\centering
\caption{PARAMETER VALUES FOR THE BATAVUS BROWSER.}
{\small
\begin{tabular}{lcl}
&&\\
\hline
Parameter & Symbol & Value \\
\hline
wheel base & $w$ & 1.121 m \\
trail & $c$ & 0.069 m \\
steer axis tilt & \emph{$\lambda$} & 0.400 rad \\
gravity & \emph{g} & 9.81 N kg$^{-1}$ \\
forward speed & \emph{v} & various m s$^{-1}$ \\

\multicolumn{3}{l}{\textbf{R}\emph{ear wheel R}}\\
radius & \emph{$r_\mathrm{R}$} & 0.341 m\\
mass & \emph{$m_\mathrm{R}$} &  3.11 kg \\
mass moments of inertia & \emph{$(I_{\mathrm{R}xx},\
I_{\mathrm{R}yy})$} &
(0.088,\ 0.152) $\mathrm{kg\ m}^2$ \\

\multicolumn{3}{l}{\emph{rear} \textbf{B}\emph{ody and frame B}}\\
position center of mass & \emph{$(x_\mathrm{B},\ z_\mathrm{B})$} &
($0.276$,\ $-0.538$) m\\
mass & \emph{$m_\mathrm{B}$} & 9.86 kg \\
mass moments of inertia & $\left[ \begin{array}{ccc}
I_{\mathrm{B}xx} &  0 & I_{\mathrm{B}xz}\\
0 & I_{\mathrm{B}yy} & 0 \\
I_{\mathrm{B}xz} & 0 & I_{\mathrm{B}zz}
\end{array} \right] $ &
$\left[ \begin{array}{ccc}
0.527 &  0 & -0.114\\
0 & 1.317 & 0 \\
-0.114 & 0 & 0.759
\end{array} \right] \ \mathrm{kg\ m}^{2}$\\

\multicolumn{3}{l}{\emph{front} \textbf{H}\emph{andlebar and fork assembly H}}\\
position center of mass & \emph{$(x_\mathrm{H},\ z_\mathrm{H})$} &
($0.867$,\ $-0.748$) m\\
mass & \emph{$m_\mathrm{H}$} & 3.22 kg \\
mass moments of inertia & $\left[ \begin{array}{ccc}
I_{\mathrm{H}xx} &  0 & I_{\mathrm{H}xz}\\
0 & I_{\mathrm{H}yy} & 0 \\
I_{\mathrm{H}xz} & 0 & I_{\mathrm{H}zz}
\end{array} \right] $ &
$\left[ \begin{array}{ccc}
    0.253  &       0  & -0.072\\
         0  & 0.246   &  0\\
   -0.072  &       0  &  0.096
\end{array} \right] \ \mathrm{kg\ m}^{2}$\\

\multicolumn{3}{l}{\textbf{F}\emph{ront wheel F}}\\
radius & \emph{$r_\mathrm{F}$} & 0.344 m\\
mass & \emph{$m_\mathrm{F}$} & 2.02 kg \\
mass moments of inertia & \emph{$(I_{\mathrm{F}xx},\
I_{\mathrm{F}yy})$} &
(0.090,\ 0.149) $\mathrm{kg\ m}^2$ \\
\hline
\end{tabular}
}
\label{tab:par}
\end{table*}

\bibliographystyle{acm}
\bibliography{bicycle}
\end{document}
