\documentclass{bmd2010a}

\begin{document}

\begin{flushleft}
{\fontsize{16pt}{20pt}\selectfont%
  An Accurate Method of Measuring a Bicycle's Physical Parameters\\}
{\fontsize{16pt}{20pt}\selectfont%
  Motorcycle Dynamics Symposium\\}
\end{flushleft}

%%%%%%%%%%%%%%%% authors %%%%%%%%%%%%%%%
\begin{flushleft}
  {\fontsize{12}{14}{Jason K. Moore}\\}
  \textit{Mechanical and Aerospace Engineering\\
          University of California, Davis\\
          One Shields Avenue, Davis, CA 95616, USA\\
          e-mail: jkmoor@ucdavis.edu
}\vspace{10pt}\\
  {\fontsize{12}{14}{Mont Hubbard}\\}
  \textit{Affiliation}\vspace{10pt}\\
  {\fontsize{12}{14}{D. E. Author}\\}
  \textit{Affiliation}\vspace{10pt}\\
\end{flushleft}

\section*{Abstract}
Accurate measurments of a bicycle's physical parameters are required for
realistic dynamic simulations. For the most basic models the geometry, mass,
mass location and mass distributions must be measured. More complex models
require measurements of tire characteristics, stiffness, damping, etc. This
paper concerns itself with the measurement of bicycle parameters required for
simulations of the benchmark bicycle presented in \cite{Meijaard2007}. This
bicycle model is made of four rigid bodies, has ideal rolling, frictionless joints,
and is assumed to be symmetric about the frame center plane. A set of 25
physical parameters are used to describe the geometry, mass, mass location and
mass distribution of each of the rigid bodies. These parameters are presented
relative to a reference frame when the bicycle is in the nominal upright
configuration. The methods described herein are based primarily off of the work
done in \cite{Kooijman2006} but have been refined for improved accuracy and
methodology. \cite{Roland1971} measured an experimental bicycle frame in a
similar fashion and both \cite{Dohring1953} and \cite{Singh1971} have used
similar techniques with scooters. We measure the characteristics of six
different bicycles, two of which are setup in two different configurations, for
a total of eight different parameter sets that can be used with, but not
limited, to the benchmark bicycle model. The parameters set is also complete
for non-linear benchmark bicycle model.

The bicycles were chosen for both variety and convience. The six bicycles are as follows:
\begin{description}
    \item[Batavus Browser] A Dutch style city bicycle. The bicycle was measured
        in two configurations. One with the instrumentation described in
        \cite{Kooijman2009a} and one as from the manufacturer.
    \item[Batavus Stratos Deluxe] A Dutch style city bicycle
    \item[Batavus Crescendo Deluxe] A Dutch style city bicycle
    \item[Gary Fisher Mountain Bike] A hardtail mountain bicycle
    \item[Bianchi Pista] A modern steel frame track racing bicycle
    \item[Yellow Bicycle] A stripped down aluminum frame road bicycle. This
        bike was measured in two configurations, the second being the fork was
        rotated in the headtube 180 degrees for larger trail.
\end{description}
\subsection{Geometry}
The benchmark bicycle needs five basic geometric measurements: the wheelbase
$w$, steer axis tilt $\lambda$, trail $c$, and the wheel radii $r_\mathrm{F}$
and $r_\mathrm{R}$. The wheelbase and steer axis tilt were measured directly
with a scale and a digital level, respectively. The trail and wheel radii were
measured indirectly by fork offset and loaded wheel circumference.
\subsection{Mass}
The mass of each of the rigid bodies were measured with a precision scale.
\subsection{Mass location}
The center's of mass of each of the rigid bodies were measured by hanging the
rigid bodies through their centers of mass and assuming symmetry. The
orientation angle of the headtube angle and the point of suspension were
measured and the center's of mass of the frame and fork were calculated from
these values. The wheel centers of mass were assumed to be in their ideal
geometric center.
\subsection{Mass Distribution}
The moment's of inertia of the rigid bodies were measured with both a torsional
pendulum and a compound pendulum.
\bibliographystyle{acm}
\bibliography{bicycle}
\end{document}
