\documentclass{article}
\usepackage{hyperref}
\usepackage{graphicx}

\begin{document}
\section{Bicycle}

\subsection{Geometry}
\subsubsection{Wheel radius}
The radius of the front $r_\mathrm{F}$ and rear $r_\mathrm{R}$ wheels were measured by measuring the linear distance traversed along the ground through eitehr 13 or 14 rotations of the wheel. Each wheel was measured separatley and the measurements were taken with a 72kg rider seated on the bicycle. A 30 meter tape measure (resolution: 2mm) was taped on a flat level smooth floor. The tire was marked with chalk and aligned with the tape measure. The accuracy of the distance measurement is approximately $\pm0.005$m. The tires were pump the recommended inflation pressure before the measurments. The wheel radius is calculated by
\begin{equation}
	r=\frac{\textrm{distance}}{2\cdot\pi\cdot\textrm{rotations}}
	\label{eq:wheelRadius}
\end{equation}
\begin{figure}[tb]
	\begin{center}
		\includegraphics[width=4in]{tireChalk.jpg}
	\end{center}
	\caption{Wheel and tire with chalk mark aligned to the tape measure.}
	\label{fig:tireChalk}
\end{figure}
\begin{table}
	\begin{tabular}{llllllll}
	Bicycle   & B      & B*     & C      & G      & P      & S      & Y \& Y*\\
	Front\\
	Pressure  & 3.5    & 3.5    & 4      & 4      & 6.9    & 4.5    & 4      \\
	Distance  & 28.060 & 27.980 & 27.980 & 29.044 & 29.366 & 27.772 & 27.925 \\
	Rotations & 13     & 13     & 13     & 14     & 14     & 13     & 13     \\
	Radius    & 0.344  & 0.343  & 0.343  & 0.330  & 0.334  & 0.340  & 0.342  \\
	Rear\\
	Pressure  & 3.5    & 3.5    & 4      & 4      & 6.9    & 4.5    & 3.5    \\
	Distance  & 27.850 & 27.835 & 27.768 & 29.788 & 29.212 & 29.774 & 27.884 \\
	Rotations & 13     & 13     & 13     & 14     & 14     & 14     & 13     \\
	Radius    & 0.341  & 0.341  & 0.340  & 0.339  & 0.332  & 0.338  & 0.341	
	\end{tabular}
	\caption{}
	\label{tab:wheelRadius}
\end{table}
\subsubsection{Head tube angle}
The head tube angle was measured directly using an electronic level with a $\pm0.2^{\circ}$ accuracy. The bicycle was fixed vertically, the front wheel was aligned with the rear frame and the bicycle was unloaded. The steer axis tilt $\lambda$ is the complement to the head tube angle.
\begin{equation}
	\lambda=\frac{\pi}{180}(90^{\circ}-\lambda_{ht})
\label{eq:headTubeAngle}
\end{equation}
\begin{table}
	\begin{tabular}{lllllllll}
	Bicycle   & B      & B*     & C      & G      & P      & S      & Y      & Y*\\
	Head tube angle & 67.1 & 67.1 & 69.0 & 71.1 & 74.2 & 73.1 & 72.7 & 70.6\\
	Steer axis tilt & 0.400 & 0.400 & 0.367 & 0.330 & 0.276 & 0.295 & 0.302 & 0.339
	\end{tabular}
	\caption{Head tube angle and steer axis tilt}
	\label{tab:lambda}
\end{table}

\subsubsection{Trail}
Trail is difficult to measure directly so we instead chose to measure the fork offset. The fork offset was measured by clamping the steer tube of the front fork into a v-block on a flat table. A ruler was used to measure the height of the center of the head tube and the height of the center of the axle axis. The fork blades where aligned such that the axle axis was parallel to the table surface.
\begin{equation}
	c=\frac{r_\mathrm{F}\sin{\lambda}-f_o}{\cos{\lambda}}
	\label{eq:trail}
\end{equation}
\begin{table}
	\begin{tabular}{lllllllll}
	Bicycle     & B     & B*    & C     & G     & P     & S     & Y     & Y*\\
	Fork offset & 0.071 & 0.071 & 0.045 & 0.039 & 0.032 & 0.045 & 0.057 & -0.057\\
	Trail       & 0.069 & 0.068 & 0.083 & 0.072 & 0.062 & 0.056 & 0.047 & 0.180
	\end{tabular}
	\caption{Head tube angle and steer axis tilt}
	\label{tab:trail}
\end{table}
\subsubsection{Wheelbase}
The wheelbase was measured directly using a tape measure while the bicycle was in the same uprigth fixed position used when measuring the head tube angle.
\begin{table}
	\begin{tabular}{lllllllll}
	Bicycle    & B     & B*    & C     & G     & P     & S     & Y     & Y*\\
	Wheel base & 1.121 & 1.121 & 1.101 & 1.070 & 0.989 & 1.037 & 1.089 & 0.985
	\end{tabular}
	\caption{Wheelbase values.}
	\label{tab:wheelbase}
\end{table}

\subsection{Mass}
The total mass of each bicycle was measured using a spring scale with a resolution of 100g. Then each of the four bicycle parts were measured using a Molen 20kg scale with a resulution of 20g.
\begin{table}
	\begin{tabular}{lllllllll}
		Bicycle     & B     & B*    & C     & G     & P    & S     & Y     & Y*\\
		total       & 18.5  & 23.5  & 20.9  & 10.5  & 9.9  & 17.6  & 10.2  & 10.2\\
		frame       & 9.86  & 14.71 & 9.18  & 4.48  & 4.49 & 7.22  & 3.31  & 3.31\\
		fork        &       & 3.22  & 4.57  & 2.52  & 2.27 & 3.04  & 2.45  & 2.45\\
		rear wheel  & 3.11  & 3.11  & 3.96  & 1.94  & 1.38 & 3.96  & 2.57  & 2.57\\
		front wheel & 2.02  & 2.02  & 3.55  & 1.50  & 1.58 & 3.33  & 1.90  & 1.90\\
		total sum   & 18.21 & 23.06 & 21.26 & 10.44 & 9.72 & 17.55 & 10.23 & 10.23
	\end{tabular}
	\caption{Mass of bicycles and parts}
	\label{tab:mass}
\end{table}
\subsection{Center of Mass}
The centers of mass of the wheels are assumed to be at their geometrical centers.
\subsubsection{Rear frame}
The rear frame was hung in three orientations from a torsional pendulum. The frame could rotate about a joint such that gravity aligned the center of mass with the pendulum axis. The new orientaion angle of the headtube $\alpha$ was measured using the digital level. A string was aligned with the pendulum axis and allowed to pass by the frame. The horizontal distance $d$ between the rear axle and the string (CoM line) was measured by aligning a ruler pendicular to the string. The distance $d$ was negative if the string fell to the right of the rear axle and positive if it fell to the left of the rear axle. These measurements allow for the calucation of the center of mass location in the global reference frame.
The frame rotation angle $\beta$ is defined as rotation of the frame in the nominal configuration to the hanging orientation, rotated about the $Y$ axis.
\begin{equation}
	\beta=\alpha-\lambda-\frac{\pi}{2}
\label{eq:frameRotAng}
\end{equation}
The CoM line is simply a line in the global reference frame with a slope $m$ and a z-intercept $b$. The slope can be shown to be
\begin{equation}
	m=\tan{\left(\beta-\frac{\pi}{2}\right)}
\label{eq:slope}
\end{equation}
The z-intercept can be shown to be
\begin{equation}
	b=-\frac{d}{\sin{\beta}}-r_\mathrm{R}
\label{eq:zInt}
\end{equation}
The CoM location can be calculated by finding the intersection of these three lines. The system is over determined so the least squares method is used to find a unique solution.
\begin{equation}
	\left[
	\begin{array}{cc}
		-m_1 & 1\\
		-m_2 & 1\\
		-m_3 & 1
	\end{array}
	\right]
	\left[
	\begin{array}{c}
		x\\
		z
	\end{array}
	\right]
	=
	\left[
	\begin{array}{c}
		b_1\\
		b_2\\
		b_3
	\end{array}
	\right]	
\label{eq:solveSlopeInt}
\end{equation}
\subsection{Moment of Inertia}
\section{Human}
\appendix
\subsection{Bicycle Descriptions}
\subsubsection{Yellow Bicycle}
The yellow is a bicycle used in the lab to demonstrate that a bicycle is stable at certain speeds. It is an aluminum road frame of unknown make with the majority of components removed. The wheels, drop handlebar, seat, seat post and bottom bracket are the only parts on the bike. The fork has been reversed to ''increase`` the stability of the bicycle. The bicycle was measured with both the fork in normal position and the fork reversed.
\begin{table}
	\begin{tabular}{ll}
		Style & Road\\
		Size & \\
		Wheels & 700c Aluminum Rims\\
		Tires & 28 x 1 3/8 x 1 5/8\\
		Sprocket & 6 speed free wheel\\
		Handlebars & Drop\\
	\end{tabular}
	\caption{}
	\label{yellowBikeSpecs}
\end{table}

\subsubsection{Batavus Browser}
The Batavus Browser is lower priced Dutch city bike. We measured the physical properities of the stock Browser model and also the same bike with the instrumentation used in our experiments.
\begin{table}
	\begin{tabular}{ll}
		Price          & � 449\\
		Wheel Size     & 28''\\
		Weight         & 18.7 kg\\
		Frame Size     & 54\\
		Frame material & High Tensile Steel\\
		Fork material  & Steel\\
		Rims           & Aluminum Airline\\
    Spokes         & Stainless steel\\
		Tires          & Trax Highway, 28'' x 1 5/8 x 1 3/8\\
		Handlebar      & Steel painted\\
		Handles        & Batavus Comfort\\
		Saddle         & Selle Royal 8274\\
		Dynamo         & Basta Trio\\
		Headlight      & Batavus Logic Light\\
		Rearlight      & Battery light\\
		Pedals         & Batavus plastic antislip\\
		Final          & Trelock RS42\\
		Theft chip     & Present\\
		Crank          & Aluminum\\
		Rack           & Steel\\
		Quick Binder   & Binder Trio\\
		Mudguards      & Steel\\
		Bell           & Kenlight\\
	\end{tabular}
	\caption{2008 Batavus Browser Specifications}
	\label{browserSpecs}
\end{table}

\subsubsection{2007 Bianchi Pista}
The Pista is a steel track bicycle. Specifications, taken from the Bianchi website~\cite{Bianchi2009}, are as follows:
\begin{table}
	\begin{tabular}{ll}
		Style            & Track Bike\\
  	Size             & 57cm\\
  	Color            & Gang Green\\
  	Frame            & Bianchi DB CrMo, rear entry track dropouts\\
  	Fork             & CrMo\\
  	Retail Price     & \$579.99\\
		Headset          & VP AheadSet, 1''threadless\\
  	Handlebar        & Bianchi steel track, 26.0mm\\
		Stem             & Bianchi alloy\\
    Brakes/Levers    & Shimano Tiagra front brake/lever\\
		Crankset         & Truvativ Touro 1.1, 48T\\
		Bottom Bracket   & Truvativ Power Spline cartridge\\
		Chain            & KMC\\
		Sprocket         & 16T fixed cog \& 16T Freewheel\\
		Pedals           & Campus pedals\\
		Wheels           & Bianchi Alex Solo wheelset, 28h rims\\
    Tires            & Continental UltraSport, 700x23C\\
    Saddle           & Bianchi Velo\\
		Seatpost         & Bianchi alloy, 27.2mm\\
		Size             & 57cm\\ 
		Seat Tube        & 570mm\\
		Top Tube-actual  & 560mm\\
		Top Tube-virtual & 560mm\\
		Chainstay        & 383mm\\
		Fork Rake        & 28mm\\
		Head Tube Angle  & 74.5$^{\circ}$\\
		Seat Tube Angle  & 75.5$^{\circ}$\\
		Wheelbase        & 967mm\\
		Standover Height & 32''  
	\end{tabular}
	\caption{}
	\label{pistaSpecs}
\end{table}
\subsubsection{Gary Fisher Mountain Bike}
\subsubsection{Batavus Crescendo Deluxe}
	
The Batavus Crescendo Deluxe is a contemporary of all-round cycling in cool black and white. A great bike for a good end to touring. And that works fine with 8 gears, suspension fork and suspension seatpost.
\begin{table}
	\begin{tabular}{ll}
		Price             & from � 869\\
		Wheel Size        & 28''\\
		Weight            & 21.5 kg\\
		Frame Size        & 53\\
		Performance/speed & Shimano Nexus 8 speed with roller brakes\\
		Color             & White / black\\
		Brakes            & Shimano roller brakes (BR-IM50)\\
		Frame type        & Ladies mono tube frame Gentlemen trend\\
		Frame material    & Aluminum 7005\\
		Front Fork        & spring, Batavus\\
		Rims              & Aluminum Rodi VR19\\
		Spokes            & Stainless steel (adapted spoke pattern)\\
		Tires             & CST End\\
		Handlebar         & Steel aluminum look\\
		Stem              & Manually adjustable, Batavus Ergo Matic III\\
		Handles           & adjustable Batavus Ergo Grip II\\
		Saddle            & Selle Royal 5160\\
		Seatpost          & spring, Post Modern Compact\\
		Dynamo            & Shimano Hub Dynamo\\
		Headlight         & Trelock Hi-Power automatic halogen\\
		Rear              & Smart battery automatically\\
		Pedals            & aluminum antislip\\
		Final							& AXA Defender\\
		Chip theft        & Present\\
		Crank             & Aluminum\\
		Rack              & Aluminum\\
		Quick Binder      & Binder Quattro\\
		Mudguards         & Batavus hyalite II\\
		Bell              & Bibia Touring\\
		Balhoofdset       & VP semi-integrated, oversized\\
		Shifters          & Batavus (8-speed)\\
		Saddlebag         & Batavus saddlebag with plakset\\
	\end{tabular}
	\caption{}
	\label{crescendoSpecs}
\end{table}
\subsection{Batavus Stratos Deluxe}
\begin{table}
	\begin{tabular}{ll}
		Switching           & Spin System (shifter)\\
		Monochromatic frame & Black\\
		Chain guard         & Semi-open\\
		Mudguards           & Aluminum\\
		Rack                & Yes\\
		Extra drive         & Regardless Motori degree\\
		Type                & Men's\\
		Frame Material      & Aluminum\\
		Rim Material        & Aluminum\\
		Rear braking system & Roller brake\\
		Front braking system & Roller Brake\\
		Number of gears      & 7 Gears\\
		Gear System          & Shimano Nexus\\
		Wheel Size           & 28 inch\\
		Tirewidth            & 37 mm\\
		Price                & 749,00\\
	\end{tabular}
	\caption{}
	\label{stratosSpecs}
\end{table}
\bibliographystyle{plain}
\bibliography{bicycle}
\end{document}